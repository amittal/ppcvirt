\documentclass[10pt,twocolumn]{article}

\usepackage{times}
\usepackage{fullpage}

\begin{document}

\title{Efficient Virtualization on Embedded Power Architecture}
\author{}
\date{}
\maketitle
\thispagestyle{empty}

\maketitle
\begin{abstract}
  Power Architecture is popular and widespread on embedded systems, and such
  platforms are
  increasingly
  being used to run virtual machines\cite{XXX}. While the Power architecture meets the
  Popek-and-Goldberg virtualization requirements for traditional trap-and-emulate
  style virtualization, the performance overhead of virtualization remains high.
  For example, workloads exhibiting a large amount of kernel activity typically
  show 3-5x slowdowns over bare-metal.

  Recent additions to the Linux kernel contain guest and host side paravirtual
  extensions for Power Architecture. While these extensions improve performance significantly, they
  are guest-intrusive, non-portable and cover only a subset of all possible
  virtualization optimizations.

  We present a set of host-side optimizations that achieve comparable
  or better performance
  than the aforementioned paravirtual extensions, yet being guest-agnostic. Our
  prototype implementation based on Linux/KVM can run (largely) unmodified, untrusted
  Power Architecture guests at near native performance for almost all workloads. The only
  guest modification required is to include a device driver for our virtual
  device which creates a shared address space between the guest and the host.
  After our modifications, KVM can boot a Linux guest 3x faster. Our solution
  provides equivalent performance (sometimes outperforming by up to XX\%) to
  the paravirtual approach previously included in the Linux kernel, without
  being guest-specific.
\end{abstract}
\section{Introduction}
Power Architecture is a dominant platform for embedded devices for it's
favourable power/performance characteristics. Virtualization on these platforms is
compelling for several applications\cite{XXX}. While newer Power platforms
have explicit support for efficient virtualization\cite{XXX}, a majority of
prevalent embedded devices run on older (and cheaper) Power platforms that use
traditional trap-and-emulate style virtualization\cite{XXX}. Hence, efficient
virtualization is highly desirable on these platforms.

The current virtualization approach on Power Architecture uses traditional
trap-and-emulate. The guest operating system is run unprivileged, causing each
execution of a privileged
operation to exit into the hypervisor. For guest workloads executing a large number
of priviliged instructions, these exits are a major performance
bottleneck. Table~\ref{tab:kvm_performance} lists the performance of vanilla Linux/KVM
on some common workloads.

The poor performance of simple trap-and-emulate style virtualization has led to
the inclusion of paravirtual extensions to the Linux kernel on both
guest and host sides. The paravirtual extension in the guest rewrites the guest kernel
code at startup time to replace most privileged instructions with
hypervisor-aware unprivileged counterparts. The unprivileged version of an
otherwise privileged guest operation typically reads or writes to a hypervisor-visible
shared address, thus providing an efficient mechanism for a {\em hypercall}.
On the host side, a mechanism is provided to setup a shared address space with the guest,
and to execute hypercalls on behalf of the guest. The third column in
Table~\ref{tab:kvm_performance} lists the performance of Linux/KVM with paravirtual
extensions. In this case, both guest and host are running Linux with their respective
paravirtual extensions. The performance improves significantly for many common
workloads.

There are some shortcomings to the paravirtual approach. Firstly, extensive guest
modifications are required making it highly specific to Linux. A different
guest OS, or subsequent updates to Linux, require substantial software
development and maintenance effort. Secondly, all guest
privileged instructions are rewritten only at kernel load time. Hence,
optimizations are
not possible for privileged code running as loadable kernel modules. This
approach will also break in the presence of dynamically generated code, and makes
assumptions about the programming discipline in Linux. For example, this approach
assumes that Linux kernel will never read it's own code as data. Violation of
these programming disciplines in future Linux versions, can render these paravirtual
extensions unusable.

We propose a host-side adaptive binary translation mechanism to optimize guest
privileged instructions at runtime. Our approach requires almost no guest modifications
(except a virtual device driver to setup shared address space), and achieves
comparable or better performance than paravirtual approaches. Because we translate
guest code at runtime (and not at kernel startup time, as done in the paravirtual
approach), we
can optimize code in both the main kernel and the loadable modules. Our approach also
does not assume anything about the guest, and ports seamlessly to different
guests or different versions of the same guest.
The last column in Table~\ref{tab:kvm_performance} summarizes the performance results of
our host-side binary translation approach. In the rest of the paper, we discuss
our approach and the various optimizations we used to achieve these results.

We translate the guest's instructions {\em in situ}, thus modifying the guest's
address space. This is in contrast with previous full binary translation approaches
that translate the entire guest
code (e.g., VMware's x86-based VMM\cite{agesen:comparison}). Our approach is much
simpler and thus incurs significantly less overhead. For example, a full binary
translator requires a table lookup on every indirect branch (e.g., function return) which
can result in high overheads on many common workloads. In contrast, our lightweight
binary translator simply translates the privileged instructions and allows all other
instructions to run natively.

There are many interesting problems we solved during the development of our
binary translation solution. 
Firstly, while modifying guest instructions, we must ensure correctness in presence
of arbitrary jumps to any instruction. We do this by relying on the fixed-length
word-aligned nature of Power Architecture instructions. Guest instructions
that need to be translated to multiple host instructions for correct emulation, are
handled by jumping to code fragments in a VMM-managed translation cache (full
details in Section~\ref{sec:binarytranslation}.

Secondly, modifying guest instructions {\em in situ} requires the VMM to maintain guest
fidelity if the guest reads/writes it's own code. We maintain fidelity by 
marking the pages containing the modified instructions {\em execute-only}. This
causes the hardware to trap into the VMM on any guest read/write access to the
modified page.

Thirdly, marking the modified pages execute-only can result in a large number of
page faults, especially due to false sharing. The problem is particularly
severe in the embedded Power Architecture which has a software-managed TLB with
a small number of variable-sized TLB entries. We found that our modifications to
the guest instructions resulted in a large number of such page faults, thus causing
another source of performance degradation. We discuss two schemes to correctly
and efficiently deal with this problem (Section~\ref{sec:reducing_page_faults}).

We rely heavily on some Power Architecture features to implement our
scheme correctly and efficiently. For example, we use
Power Architecture's {\tt read-write-execute} page protection bits to ensure guest
fidelity. In contrast, the x86 architecture provides only {\tt read-only} and
{\tt no-execute} (NX) bits, which are less powerful, and insufficient
to implement our scheme. As another
example, we use Power
Architecture's fixed-length aligned instruction encoding to be able to
perform in-place binary translation correctly. Such ideas cannot be used
on x86 where the guest could potentially jump to the middle of an instruction.
We find that these subtle
differences in different architectures greatly impact VMM design. Just like some
of our aforementioned ideas are not applicable to
x86, some of the ideas in VMware's x86-based binary translator
are not applicable to Power Architecture. For example, VMware's binary translator
relies on x86's segmentation hardware but Power Architecture does not
support segmentation.

In summary, this paper presents an efficient host-side optimization solution for
Power virtualization. Our approach is based on in-situ binary translation.
The only guest-side modification required by our solution is a virtual
device driver to map a shared address space between the guest and the host.
We can run untrusted guests and preserve guest fidelity by implementing efficient
tracing mechanisms.
The paper is organized as
follows. Section~\ref{sec:performance_char} characterizes the performance of
KVM on Power Architecture and discusses the typical sources of overhead.
Section~\ref{sec:bintrans} discusses our in-situ binary translation approach.
Section~\ref{sec:tracing} discusses our tracing mechanism to maintain guest fidelity.
We discuss the performance results obtained by
implementing these mechanisms in Section~\ref{sec:results1}. Section~\ref{sec:tracingopt}
discusses further optimizations to make the tracing mechanism more efficient.
Section~\ref{sec:results2} presents our final performance results, and
Section~\ref{sec:conclusion} concludes.

\section{Performance Characterization}
Discuss the performance of base KVM and analyze it (Table \ref{tab:PerfBase}). Talk about the number of exits
resulting from each type of privileged instruction (Table \ref{tab:NumExitsBase}). Also, talk about the number
of exits resulting from each PC value (Table \ref{tab:NumPCBase}).

\begin{table}[!b]
\centering
\caption{Performance of Base KVM}
     \begin{tabular}{lcc} \hline
       Benchmark  & Performance (\% of native) \\ \hline
       Linux Boot & 21.7 \\
       Echo Spawn & 6.4 \\
       Find & ${-}$ \\
       \hline
     \end{tabular}
\label{tab:PerfBase}
\end{table}


\begin{table}[!b]
\centering
\caption{Sources of VM exits for linux boot}
     \begin{tabular}{lcc} \hline
       Instruction class  & Exit count & \% of total exits  \\ \hline
       MFSPR & 4484245 & 33.8  \\
       WRTEE & 2792109 & 21.1  \\
       MTSPR & 2307647 & 17.4  \\
       RFI & 575302 & 9.5 \\
       MTMSR & 413847 & 4.3 \\
       TLBWE & 391813 & 3.1 \\
       DTLBREAL & 247282 & 2.9 \\
       ITLBREAL & 241070 & 1.8 \\
       DTLBVIRT & 198239 & 1.5 \\
       ITLBVIRT & 192046 & 1.4 \\
       \hline
     \end{tabular}
\label{tab:NumExitsBase}
\end{table}

\begin{table}[!b]
\centering
\caption{PCs responsible for VM exits for linux boot}
     \begin{tabular}{lcc} \hline
       Exits count  & PC count & \% of total exits  \\ \hline
       $>$20000 & 93 & 91.9  \\
       $>$10000 & 23 & 3.1  \\
       $>$5000 & 68 & 4.2  \\
       $>$2000 & 12 & 0.3 \\
       $>$1000 & 17 & 0.2 \\
       $<$1000 & 299 & 0.2 \\
       \hline
     \end{tabular}
\label{tab:NumPCBase}
\end{table}
\section{Lightweight Adaptive Binary Translation}
Discuss our BT implementation. Discuss the different opcodes and their replaced
counterparts
\subsection{Shared Address Space}
Discuss how a shared address space is setup between the guest and the host.

\section{Maintaining Guest Fidelity through Read/Write Tracing}
Discuss our tracing mechanism.

\section{Experimental Results - I}
Discuss the experimental results obtained after implementing binary translation
and tracing. Discuss the sources of overhead, and TLB thrashing.

\section{Further Optimizations}
\subsection{Adaptive TLB splitting/merging}
\subsection{Optimization of DSI reads} - explaination - tracing of root cause by studying results of nanobenchmark syscall - comparison of results with DM-read

\section{Experimental Results - II}
Compare with base KVM, paravirtual, and baremetal. Discuss sources of overhead. Discuss
why performance has improved.

\section{Comparison with known x86-based binary translation approach}
Compare with VMware's x86-based binary translation approach.

\section{Conclusion}

\bibliography{millirollbacks}
\bibliographystyle{abbrv}

\end{document}
